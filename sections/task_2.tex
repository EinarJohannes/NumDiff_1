\section{Theory}
%Random theorem with proof:
\begin{thm}
    For any real number $x$, $x^2 \geq 0$. Equality holds if and only if $x=0$.
\end{thm}
%Proof of theorem:
\begin{proof}
    If $x=0$, then $x^2=0$. If $x\neq 0$, then $x^2 > 0$. This proves the theorem.
    Gjør endring i beviset for å sjekke om det funker.
    Gaute gjør endring for å se om det funker
\end{proof}

%pythagoras theorem
\begin{thm}
    The Pythagorean theorem states that for any right triangle with sides of length $a$, $b$ and $c$ where $c$ is the hypotenuse, the following equation holds:
    \begin{equation}
        a^2 + b^2 = c^2
    \end{equation}
\end{thm}
%proof of pythagoras theorem
\begin{proof}
    The Pythagorean theorem states that for any right triangle with sides of length $a$, $b$ and $c$ where $c$ is the hypotenuse, the following equation holds:
    \begin{equation}
        a^2 + b^2 = c^2
    \end{equation}
    This can be proven by considering the area of the square with side length $a+b$ in two different ways. The area of the square is $(a+b)^2 = a^2 + 2ab + b^2$. The area can also be calculated by adding the areas of the squares with side lengths $a$ and $b$ to get $a^2 + b^2 + 2ab$. Since the two expressions for the area are equal, we get the Pythagorean theorem.
\end{proof}
